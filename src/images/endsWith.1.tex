\documentclass[12pt]{standalone}

\usepackage{tikz}
\usetikzlibrary{matrix}
\usetikzlibrary{calc}
\usetikzlibrary{positioning}
\usetikzlibrary{intersections}
\usetikzlibrary{decorations}
\usetikzlibrary{decorations.pathmorphing}
\usetikzlibrary{decorations.pathreplacing}
\usetikzlibrary{decorations.text}
\usetikzlibrary{arrows.meta}
\usetikzlibrary{backgrounds}
\usetikzlibrary{shapes.multipart}
\usetikzlibrary{shapes.symbols}
\usetikzlibrary{shapes.arrows}

\usepackage{ifthen}
\usepackage{calc}
\usepackage{inconsolata}
\usepackage{pifont} % for true and false symbols

\makeatletter
\makeatletter

\pgfkeys{
    /function/.cd,
    north port/.initial=1,
    south port/.initial=1,
    east port/.initial=0,
    west port/.initial=0,
    port width/.initial=3mm,
    port sep/.initial=1mm,
    north port width/.initial=\expandafter\pgfkeysvalueof{/function/port width},
    north port sep/.initial=\expandafter\pgfkeysvalueof{/function/port sep},
    south port width/.initial=\expandafter\pgfkeysvalueof{/function/port width},
    south port sep/.initial=\expandafter\pgfkeysvalueof{/function/port sep},
    east port height/.initial=\expandafter\pgfkeysvalueof{/function/port width},
    east port sep/.initial=\expandafter\pgfkeysvalueof{/function/port sep},
    west port height/.initial=\expandafter\pgfkeysvalueof{/function/port width},
    west port sep/.initial=\expandafter\pgfkeysvalueof{/function/port sep},
}

\def\port#1{
    \pgfpathlineto{\pgfqpoint{\z@}{\z@}}
    \pgfpathlineto{\pgfqpoint{.5mm}{\z@}}
    \pgfpathlineto{\pgfqpoint{.5mm}{.5mm}}
    \pgfpathlineto{\pgfqpoint{\z@}{1mm}}
    \pgfpathlineto{\pgfqpoint{\z@}{1.5mm}}
    \pgfpathlineto{\pgfqpoint{#1}{1.5mm}}
    \pgfpathlineto{\pgfqpoint{#1}{1mm}}
    \pgfpathlineto{\pgfpoint{#1 - .5mm}{.5mm}}
    \pgfpathlineto{\pgfpoint{#1 - .5mm}{\z@}}
    \pgfpathlineto{\pgfqpoint{#1}{\z@}}
}

\pgfdeclareshape{function}{%
    \inheritsavedanchors[from=rectangle]
    \inheritanchor[from=rectangle]{north east}
    \inheritanchor[from=rectangle]{north west}
    \inheritanchor[from=rectangle]{south east}
    \inheritanchor[from=rectangle]{south west}
    \savedanchor\southwest{%
        % box width
        \pgf@x=\the\wd\pgfnodeparttextbox
        % + inner sep
        \setlength\pgf@xa{\pgfshapeinnerxsep}
        \advance\pgf@x by 2\pgf@xa%
        % is width < min width?
        \setlength\pgf@xa{\pgfshapeminwidth}
        \ifdim\pgf@x<\pgf@xa \pgf@x=\pgf@xa \fi
        % is width < north port width
        \pgfmathsetlength\@tempdima{\pgfkeysvalueof{/function/north port width} + \pgfkeysvalueof{/function/north port sep}}
        \pgfmathsetlength\pgf@xa{\pgfkeysvalueof{/function/north port} * \@tempdima}
        \ifdim\pgf@x<\pgf@xa \pgf@x=\pgf@xa \fi
        % is width < south port width
        \pgfmathsetlength\@tempdima{\pgfkeysvalueof{/function/south port width} + \pgfkeysvalueof{/function/south port sep}}
        \pgfmathsetlength\pgf@xa{\pgfkeysvalueof{/function/south port} * \@tempdima}
        \ifdim\pgf@x<\pgf@xa \pgf@x=\pgf@xa \fi
        % + outer sep
        \setlength\pgf@xa{\pgfshapeouterxsep}%
        %
        \advance\pgf@x by 2\pgf@xa%
        \pgf@x = -.5\pgf@x
        \advance\pgf@x by .5\wd\pgfnodeparttextbox
        % box height
        \pgf@y=\the\ht\pgfnodeparttextbox
        % + inner sep
        \setlength\pgf@ya{\pgfshapeinnerysep}
        \advance\pgf@y by 2\pgf@ya%
        % is height < min height?
        \setlength\pgf@ya{\pgfshapeminheight}
        \ifdim\pgf@y<\pgf@ya \pgf@y=\pgf@ya \fi
        % is height < east port height
        \pgfmathsetlength\@tempdima{\pgfkeysvalueof{/function/east port height} + \pgfkeysvalueof{/function/east port sep}}
        \pgfmathsetlength\pgf@ya{\pgfkeysvalueof{/function/east port} * \@tempdima}
        \ifdim\pgf@y<\pgf@ya \pgf@y=\pgf@ya \fi
        % is height < west port height
        \pgfmathsetlength\@tempdima{\pgfkeysvalueof{/function/west port height} + \pgfkeysvalueof{/function/west port sep}}
        \pgfmathsetlength\pgf@ya{\pgfkeysvalueof{/function/west port} * \@tempdima}
        \ifdim\pgf@y<\pgf@ya \pgf@y=\pgf@ya \fi
        % + outer sep
        \setlength\pgf@ya{\pgfshapeouterysep}%
        %
        \advance\pgf@y by 2\pgf@ya%
        \pgf@y = -.5\pgf@y
        \advance\pgf@y by .5\ht\pgfnodeparttextbox
    }
    \savedanchor\northeast{%
        % box width
        \pgf@x=\the\wd\pgfnodeparttextbox
        % + inner sep
        \setlength\pgf@xa{\pgfshapeinnerxsep}
        \advance\pgf@x by 2\pgf@xa%
        % is width < min width?
        \setlength\pgf@xa{\pgfshapeminwidth}
        \ifdim\pgf@x<\pgf@xa \pgf@x=\pgf@xa \fi
        % is width < north port width
        \pgfmathsetlength\@tempdima{\pgfkeysvalueof{/function/north port width} + \pgfkeysvalueof{/function/north port sep}}
        \pgfmathsetlength\pgf@xa{\pgfkeysvalueof{/function/north port} * \@tempdima}
        \ifdim\pgf@x<\pgf@xa \pgf@x=\pgf@xa \fi
        % is width < south port width
        \pgfmathsetlength\@tempdima{\pgfkeysvalueof{/function/south port width} + \pgfkeysvalueof{/function/south port sep}}
        \pgfmathsetlength\pgf@xa{\pgfkeysvalueof{/function/south port} * \@tempdima}
        \ifdim\pgf@x<\pgf@xa \pgf@x=\pgf@xa \fi
        % + outer sep
        \setlength\pgf@xa{\pgfshapeouterxsep}%
        %
        \advance\pgf@x by 2\pgf@xa%
        \pgf@x = .5\pgf@x
        \advance\pgf@x by .5\wd\pgfnodeparttextbox
        %
        % box height
        \pgf@y=\the\ht\pgfnodeparttextbox
        % + inner sep
        \setlength\pgf@ya{\pgfshapeinnerysep}
        \advance\pgf@y by 2\pgf@ya%
        % is height < min height?
        \setlength\pgf@ya{\pgfshapeminheight}
        \ifdim\pgf@y<\pgf@ya \pgf@y=\pgf@ya \fi
        % is height < east port height
        \pgfmathsetlength\@tempdima{\pgfkeysvalueof{/function/east port height} + \pgfkeysvalueof{/function/east port sep}}
        \pgfmathsetlength\pgf@ya{\pgfkeysvalueof{/function/east port} * \@tempdima}
        \ifdim\pgf@y<\pgf@ya \pgf@y=\pgf@ya \fi
        % is height < west port height
        \pgfmathsetlength\@tempdima{\pgfkeysvalueof{/function/west port height} + \pgfkeysvalueof{/function/west port sep}}
        \pgfmathsetlength\pgf@ya{\pgfkeysvalueof{/function/west port} * \@tempdima}
        \ifdim\pgf@y<\pgf@ya \pgf@y=\pgf@ya \fi
        % + outer sep
        \setlength\pgf@ya{\pgfshapeouterysep}%
        %
        \advance\pgf@y by 2\pgf@ya%
        \pgf@y = .5\pgf@y
        \advance\pgf@y by .5\ht\pgfnodeparttextbox
    }
    \savedanchor\centerpoint{%
        \pgf@x = .5\wd\pgfnodeparttextbox
        \pgf@y = .5\ht\pgfnodeparttextbox
    }
    \saveddimen\functionwidth{%
        % box width
        \pgf@x=\the\wd\pgfnodeparttextbox
        % + inner sep
        \setlength\pgf@xa{\pgfshapeinnerxsep}
        \advance\pgf@x by 2\pgf@xa%
        % is width < min width?
        \setlength\pgf@xa{\pgfshapeminwidth}
        \ifdim\pgf@x<\pgf@xa \pgf@x=\pgf@xa \fi
        % is width < north port width
        \pgfmathsetlength\@tempdima{\pgfkeysvalueof{/function/north port width} + \pgfkeysvalueof{/function/north port sep}}
        \pgfmathsetlength\pgf@xa{\pgfkeysvalueof{/function/north port} * \@tempdima}
        \ifdim\pgf@x<\pgf@xa \pgf@x=\pgf@xa \fi
        % is width < south port width
        \pgfmathsetlength\@tempdima{\pgfkeysvalueof{/function/south port width} + \pgfkeysvalueof{/function/south port sep}}
        \pgfmathsetlength\pgf@xa{\pgfkeysvalueof{/function/south port} * \@tempdima}
        \ifdim\pgf@x<\pgf@xa \pgf@x=\pgf@xa \fi
    }
    \saveddimen\functionheight{%
        % box height
        \pgf@x=\the\ht\pgfnodeparttextbox
        % + inner sep
        \setlength\pgf@xa{\pgfshapeinnerysep}
        \advance\pgf@x by 2\pgf@xa%
        % is height < min height?
        \setlength\pgf@xa{\pgfshapeminheight}
        \ifdim\pgf@x<\pgf@xa \pgf@x=\pgf@xa \fi
        % is height < east port height
        \pgfmathsetlength\@tempdima{\pgfkeysvalueof{/function/east port height} + \pgfkeysvalueof{/function/east port sep}}
        \pgfmathsetlength\pgf@xa{\pgfkeysvalueof{/function/east port} * \@tempdima}
        \ifdim\pgf@x<\pgf@xa \pgf@x=\pgf@xa \fi
        % is height < west port height
        \pgfmathsetlength\@tempdima{\pgfkeysvalueof{/function/west port height} + \pgfkeysvalueof{/function/west port sep}}
        \pgfmathsetlength\pgf@xa{\pgfkeysvalueof{/function/west port} * \@tempdima}
        \ifdim\pgf@x<\pgf@xa \pgf@x=\pgf@xa \fi
    }
    \saveddimen\functionnorthportwidth{%
        \pgfmathsetlength\@tempdima{\pgfkeysvalueof{/function/north port width} + \pgfkeysvalueof{/function/north port sep}}
        \pgfmathsetlength\pgf@x{\pgfkeysvalueof{/function/north port} * \@tempdima}
    }
    \saveddimen\functionsouthportwidth{%
        \pgfmathsetlength\@tempdima{\pgfkeysvalueof{/function/south port width} + \pgfkeysvalueof{/function/south port sep}}
        \pgfmathsetlength\pgf@x{\pgfkeysvalueof{/function/south port} * \@tempdima}
    }
    \saveddimen\functioneastportheight{%
        \pgfmathsetlength\@tempdima{\pgfkeysvalueof{/function/east port height} + \pgfkeysvalueof{/function/east port sep}}
        \pgfmathsetlength\pgf@x{\pgfkeysvalueof{/function/east port} * \@tempdima}
    }
    \saveddimen\functionwestportheight{%
        \pgfmathsetlength\@tempdima{\pgfkeysvalueof{/function/west port height} + \pgfkeysvalueof{/function/west port sep}}
        \pgfmathsetlength\pgf@x{\pgfkeysvalueof{/function/west port} * \@tempdima}
    }
    \saveddimen\northportoffset{%
        \pgf@x=\z@
        \ifnum\pgfkeysvalueof{/function/north port}>0 \advance\pgf@x by 1.5mm \fi
    }
    \saveddimen\southportoffset{%
        \pgf@x=\z@
        \ifnum\pgfkeysvalueof{/function/south port}>0 \advance\pgf@x by -1.5mm \fi
    }
    \saveddimen\eastportoffset{%
        \pgf@x=\z@
        \ifnum\pgfkeysvalueof{/function/east port}>0 \advance\pgf@x by 1.5mm \fi
    }
    \saveddimen\westportoffset{%
        \pgf@x=\z@
        \ifnum\pgfkeysvalueof{/function/west port}>0 \advance\pgf@x by -1.5mm \fi
    }
    \anchor{center}{%
        \centerpoint
    }
    \anchor{north}{%
        \northeast \@tempdimb=\pgf@x \@tempdima=\pgf@y
        \southwest 
        \pgfmathsetlength\pgf@x{(\pgf@x + \@tempdimb) / 2}
        \pgf@y=\@tempdima
        \advance\pgf@y by \northportoffset
    }
    \anchor{south}{%
        \northeast \@tempdimb=\pgf@x
        \southwest \@tempdima=\pgf@y
        \pgfmathsetlength\pgf@x{(\pgf@x + \@tempdimb) / 2}
        \pgf@y=\@tempdima
        \advance\pgf@y by \southportoffset
    }
    \anchor{east}{%
        \southwest \@tempdima=\pgf@y
        \northeast
        \pgfmathsetlength\pgf@y{(\pgf@y + \@tempdima) / 2}
        \advance\pgf@x by \eastportoffset
    }
    \anchor{west}{%
        \northeast \@tempdima=\pgf@y
        \southwest
        \pgfmathsetlength\pgf@y{(\pgf@y + \@tempdima) / 2}
        \advance\pgf@x by \westportoffset
    }

    \anchor{north param 1}{
        \southwest \pgf@xa=\pgf@x
        \northeast \pgf@x=\pgf@xa
        \pgfmathsetlength\@tempdima{\functionwidth - \functionnorthportwidth}
        \advance\pgf@x by .5\@tempdima % start of north params
        \pgfmathsetlength\@tempdima{\pgfkeysvalueof{/function/north port sep} + \pgfkeysvalueof{/function/north port width}}
        \advance\pgf@x by .5\@tempdima
        \advance\pgf@y by \northportoffset
    }
    \anchor{north param 2}{
        \southwest \pgf@xa=\pgf@x
        \northeast \pgf@x=\pgf@xa
        \pgfmathsetlength\@tempdima{\functionwidth - \functionnorthportwidth}
        \advance\pgf@x by .5\@tempdima % start of north params
        \pgfmathsetlength\@tempdima{\pgfkeysvalueof{/function/north port sep} + \pgfkeysvalueof{/function/north port width}}
        \advance\pgf@x by 1.5\@tempdima
        \advance\pgf@y by \northportoffset
    }

    \anchor{east param 1}{
        \northeast
        \pgfmathsetlength\@tempdima{\functionheight - \functioneastportheight}
        \advance\pgf@y by -.5\@tempdima % start of north params
        \pgfmathsetlength\@tempdima{\pgfkeysvalueof{/function/east port sep} + \pgfkeysvalueof{/function/east port height}}
        \advance\pgf@y by -.5\@tempdima
        \advance\pgf@x by \eastportoffset
    }
    \anchor{east param 2}{
        \northeast
        \pgfmathsetlength\@tempdima{\functionheight - \functioneastportheight}
        \advance\pgf@y by -.5\@tempdima % start of north params
        \pgfmathsetlength\@tempdima{\pgfkeysvalueof{/function/east port sep} + \pgfkeysvalueof{/function/east port height}}
        \advance\pgf@y by -1.5\@tempdima
        \advance\pgf@x by \eastportoffset
    }
    \anchor{south param 1}{
        \southwest
        \pgfmathsetlength\@tempdima{\functionwidth - \functionsouthportwidth}
        \advance\pgf@x by .5\@tempdima % start of north params
        \pgfmathsetlength\@tempdima{\pgfkeysvalueof{/function/east port sep} + \pgfkeysvalueof{/function/east port height}}
        \advance\pgf@x by .5\@tempdima
        \advance\pgf@y by \southportoffset
    }
    \anchor{south param 2}{
        \southwest
        \pgfmathsetlength\@tempdima{\functionwidth - \functionsouthportwidth}
        \advance\pgf@x by .5\@tempdima % start of north params
        \pgfmathsetlength\@tempdima{\pgfkeysvalueof{/function/east port sep} + \pgfkeysvalueof{/function/east port height}}
        \advance\pgf@x by 1.5\@tempdima
        \advance\pgf@y by \southportoffset
    }
    \anchor{west param 1}{
        \southwest \pgf@xa=\pgf@x
        \northeast \pgf@x=\pgf@xa
        \pgfmathsetlength\@tempdima{\functionheight - \functionwestportheight}
        \advance\pgf@y by -.5\@tempdima % start of north params
        \pgfmathsetlength\@tempdima{\pgfkeysvalueof{/function/west port sep} + \pgfkeysvalueof{/function/west port height}}
        \advance\pgf@y by -.5\@tempdima
        \advance\pgf@x by \westportoffset
    }
    \anchor{west param 2}{
        \southwest \pgf@xa=\pgf@x
        \northeast \pgf@x=\pgf@xa
        \pgfmathsetlength\@tempdima{\functionheight - \functionwestportheight}
        \advance\pgf@y by -.5\@tempdima % start of north params
        \pgfmathsetlength\@tempdima{\pgfkeysvalueof{/function/west port sep} + \pgfkeysvalueof{/function/west port height}}
        \advance\pgf@y by -1.5\@tempdima
        \advance\pgf@x by \westportoffset
    }

    \backgroundpath{%
        \southwest \pgf@xa=\pgf@x \pgf@ya=\pgf@y
        \northeast \pgf@xb=\pgf@x \pgf@yb=\pgf@y
        \pgf@xc=\pgf@xa
        \pgf@yc=\pgf@yb
        % north
        \pgfpathmoveto{\pgfqpoint{\pgf@xc}{\pgf@yc}}
        \ifnum\pgfkeysvalueof{/function/north port}>0
        \pgfmathsetlength\@tempdima{\functionwidth - \functionnorthportwidth}
        \advance\pgf@xc by .5\@tempdima % start of north ports
        \pgfmathsetlength\@tempdima{\pgfkeysvalueof{/function/north port sep}}
        \foreach \i in {1,...,\pgfkeysvalueof{/function/north port}}{%
            \advance\pgf@xc by .5\@tempdima
            \pgftransformshift{\pgfqpoint{\pgf@xc}{\pgf@yc}}
            \port{\pgfkeysvalueof{/function/north port width}}
            \pgftransformreset
            \global\advance\pgf@xc by \pgfkeysvalueof{/function/north port width}
            \global\advance\pgf@xc by .5\@tempdima
        }
        \fi
        % east
        \pgf@xc=\pgf@xb
        \pgf@yc=\pgf@yb
        \pgfpathlineto{\pgfqpoint{\pgf@xc}{\pgf@yc}}
        \ifnum\pgfkeysvalueof{/function/east port}>0
        \pgfmathsetlength\@tempdima{\functionheight - \functioneastportheight}
        \advance\pgf@yc by -.5\@tempdima % start of east ports
        \pgfmathsetlength\@tempdima{\pgfkeysvalueof{/function/east port sep}}
        \foreach \i in {1,...,\pgfkeysvalueof{/function/east port}}{%
            \advance\pgf@yc by -.5\@tempdima
            \pgftransformshift{\pgfqpoint{\pgf@xc}{\pgf@yc}}
            \pgftransformrotate{-90}
            \port{\pgfkeysvalueof{/function/east port height}}
            \pgftransformreset
            \global\advance\pgf@yc by -\pgfkeysvalueof{/function/east port height}
            \global\advance\pgf@yc by -.5\@tempdima
        }
        \fi
        % south
        \pgf@xc=\pgf@xb
        \pgf@yc=\pgf@ya
        \pgfpathlineto{\pgfqpoint{\pgf@xc}{\pgf@yc}}
        \ifnum\pgfkeysvalueof{/function/south port}>0
        \pgfmathsetlength\@tempdima{\functionwidth - \functionsouthportwidth}
        \advance\pgf@xc by -.5\@tempdima % start of south ports
        \pgfmathsetlength\@tempdima{\pgfkeysvalueof{/function/south port sep}}
        \foreach \i in {1,...,\pgfkeysvalueof{/function/south port}}{%
            \advance\pgf@xc by -.5\@tempdima
            \pgftransformshift{\pgfqpoint{\pgf@xc}{\pgf@yc}}
            \pgftransformscale{-1}
            \port{\pgfkeysvalueof{/function/south port width}}
            \pgftransformreset
            \global\advance\pgf@xc by -\pgfkeysvalueof{/function/south port width}
            \global\advance\pgf@xc by -.5\@tempdima
        }
        \fi
        % west
        \pgf@xc=\pgf@xa
        \pgf@yc=\pgf@ya
        \pgfpathlineto{\pgfqpoint{\pgf@xc}{\pgf@yc}}
        \ifnum\pgfkeysvalueof{/function/west port}>0
        \pgfmathsetlength\@tempdima{\functionheight - \functionwestportheight}
        \advance\pgf@yc by .5\@tempdima % start of west ports
        \pgfmathsetlength\@tempdima{\pgfkeysvalueof{/function/west port sep}}
        \foreach \i in {1,...,\pgfkeysvalueof{/function/west port}}{%
            \advance\pgf@yc by .5\@tempdima
            \pgftransformshift{\pgfqpoint{\pgf@xc}{\pgf@yc}}
            \pgftransformrotate{90}
            \port{\pgfkeysvalueof{/function/west port height}}
            \pgftransformreset
            \global\advance\pgf@yc by \pgfkeysvalueof{/function/west port height}
            \global\advance\pgf@yc by .5\@tempdima
        }
        \fi
        \pgfpathclose
        \pgfusepath{stroke}
    }
}
\makeatother

\makeatother

\begin{document}
  \begin{tikzpicture}[
    node distance=0,
    show background rectangle,
    loose background,
    background rectangle/.style={fill=white}
  ]

  \newcommand{\true}{\ifmmode $\textcolor{green}{\ding{51}}$ \else \textcolor{green}{\ding{51}} \fi}
\newcommand{\false}{\ifmmode $\textcolor{red}{\ding{55}}$ \else \textcolor{red}{\ding{55}} \fi}

\newlength{\cellwidth} \setlength{\cellwidth}{1.3cm}
\newlength{\cellheight} \setlength{\cellheight}{\cellwidth}
\newlength{\cellborderwidth} \setlength{\cellborderwidth}{0.4mm}
\newlength{\elementswidth} \setlength{\elementswidth}{1.4\cellwidth}

\tikzstyle{collection element}=[
  line width=\cellborderwidth,
  minimum width=\cellwidth,
  minimum height=\cellheight,
  outer sep=0mm,
]

\tikzstyle{collection}=[
  inner sep=0,
  column sep=-\cellborderwidth,
  row sep=-\cellborderwidth,
  text height=1.45ex,
  text depth=0.5ex,
  node distance=1mm,
  nodes={
    draw,
    collection element,
  }
]

\newcommand{\smallcolfactor}{.75}
\tikzstyle{small collection}=[
  collection,
  nodes={
    minimum width=\smallcolfactor\cellwidth,
    minimum height=\smallcolfactor\cellheight,
  },
]

\tikzstyle{empty collection}=[
  draw=gray,
  minimum height=\cellheight,
  dashed,
  line width=0.2mm,
  node contents={},
]

\tikzstyle{elements before}=[
  decorate,
  decoration={
    show path construction,
    lineto code={
      \tikzstyle{every path}=[line width=\cellborderwidth]
      \draw (\tikzinputsegmentfirst) [loosely dotted, dash pattern=on .2mm off 1mm] -- ($ (\tikzinputsegmentfirst)!.85!(\tikzinputsegmentlast) $);
      \draw ($ (\tikzinputsegmentfirst)!.85!(\tikzinputsegmentlast) $) -- (\tikzinputsegmentlast);
    }
  }
]

\tikzstyle{elements between}=[
  decorate,
  decoration={
    show path construction,
    lineto code={
      \draw [line width=\cellborderwidth]
        (\tikzinputsegmentfirst) -- ($ (\tikzinputsegmentfirst)!.15!(\tikzinputsegmentlast) $)
        ($ (\tikzinputsegmentfirst)!.85!(\tikzinputsegmentlast) $) -- (\tikzinputsegmentlast);
      \draw [loosely dotted, dash pattern=on .2mm off 1mm]
        ($ (\tikzinputsegmentfirst)!.15!(\tikzinputsegmentlast) $) -- ($ (\tikzinputsegmentfirst)!.85!(\tikzinputsegmentlast) $);
    }
  }
]

\tikzstyle{elements after}=[
  decorate,
  decoration={
    show path construction,
    lineto code={
      \draw [line width=\cellborderwidth] (\tikzinputsegmentfirst) -- ($ (\tikzinputsegmentfirst)!.15!(\tikzinputsegmentlast) $);
      \draw [loosely dotted, dash pattern=on .2mm off 1mm] ($ (\tikzinputsegmentfirst)!.15!(\tikzinputsegmentlast) $) -- (\tikzinputsegmentlast);
    }
  }
]

\newcommand{\elem}[3][] {
  \node (#2#3) [#1] {$#2_{#3}$};
}

\newcommand{\helem}[2][\elementswidth] {
  \node (x) [draw=none, minimum width=#1] {};
  \draw (x.north west) [elements #2] -- (x.north east);
  \draw (x.south west) [elements #2] -- (x.south east);
}
\newcommand{\velem}[2][\elementswidth]{
  \node (x) [draw=none, minimum height=#1] {};
  \draw (x.north west) [elements #2] -- (x.south west);
  \draw (x.north east) [elements #2] -- (x.south east);
}

\tikzstyle{map}=[
  collection,
]

\tikzstyle{maps to}=[
  decorate,
  decoration={
    show path construction,
    lineto code={
      \tikzstyle{every path}=[line width=.75\cellborderwidth]
      \fill (\tikzinputsegmentfirst) [white] circle [radius=2mm];
      \fill (\tikzinputsegmentfirst) circle [radius=1mm];
      \draw ($ (\tikzinputsegmentfirst) + (0, 2mm) $) [draw,  line width=\cellborderwidth] arc [start angle=90, end angle=270, radius=2mm];
      \draw (\tikzinputsegmentfirst) [-Triangle] -- (\tikzinputsegmentlast);
    }
  }
]

\tikzstyle{option}=[
  draw,
  line width=0.3mm,
  minimum size=1.1cm,
  outer sep=0
]

\tikzstyle{some}=[
  option,
  rectangle split,
  rectangle split parts=2,
  rectangle split draw splits=true
]

\newcommand{\some}[1] {
  \small \texttt{Some} \nodepart{two} #1
}

\tikzstyle{none}=[
  option,
  node contents={\small \texttt{None}},
]

\tikzstyle{tuple of collections}=[
  collection,
  inner sep=.25\cellheight,
  nodes={inner sep=0},
  left delimiter=(,
  right delimiter=),
]

\newcommand{\tuplecomma} {
  \node [draw=none, text height=\cellheight, minimum width=0.7cm, font=\Huge] {\,,\,\,\,};
}

\tikzstyle{function}=[
  draw,
  shape=function,
  line width=\cellborderwidth,
  minimum width=\cellwidth,
  minimum height=\cellheight,
  outer sep=0mm,
]

\tikzstyle{exception}=[
  draw,
  decorate,
  decoration={
    zigzag,
    segment length=2mm,
    amplitude=0.25mm
  },
  font=\ttfamily
]

\tikzstyle{arrow}=[
  draw=black,
  -latex,
]

\tikzstyle{reverse arrow}=[
  draw=black,
  latex-,
]

\tikzstyle{smooth}=[
  out=270,
  in=90,
]

\tikzstyle{smooth arrow}=[
  smooth,
  arrow,
]

\tikzstyle{arrow label}=[
  font=\footnotesize,
  fill=white,
]

\tikzset{horizontal bridge/.pic={
  \fill [white] circle [radius=1mm];
  \draw +(0, -1mm) -- +(0, 1mm);
  \draw +(-1mm, 0) arc [start angle=180, end angle=0, radius=1mm];
}}

\tikzset{vertical bridge/.pic={
  \fill [white] circle [radius=1mm];
  \draw +(-1mm, 0) -- +(1mm, 0);
  \draw +(0, -1mm) arc [start angle=270, end angle=90, radius=1mm];
}}

\newlength{\bigarrowwidth}
\setlength{\bigarrowwidth}{5mm}
\tikzstyle{big arrow}=[
  draw,
  line width=0.4mm,
  single arrow,
  single arrow tip angle=90,
  single arrow head extend=1mm,
  minimum width=\bigarrowwidth,
  minimum height=\bigarrowwidth,
  node contents={},
]

\tikzstyle{erase}=[
  preaction={
    draw=white,
    line cap=butt,
    -,
    line width=0.5em,
  },
  shorten <= 1pt,
  shorten >= 1pt,
]

\tikzstyle{start dots}=[
  decorate,
  decoration={
    border,
    angle=90,
    segment length=2pt,
    amplitude=\pgflinewidth,
    pre=curveto,
    pre length=0,
    post=curveto,
    post length=5mm
  }
]

\tikzstyle{middle dots}=[
  decorate,
  decoration={
    border,
    angle=90,
    segment length=2pt,
    amplitude=\pgflinewidth,
    pre=curveto,
    post=curveto,
    pre length=5mm,
    post length=5mm,
  }
]

\tikzstyle{iteration}=[
  arrow,
  dashed,
  gray!75,
]

\tikzstyle{with next label}=[
  postaction={
    decorate=true,
    decoration={
      text color=gray!75,
      text along path,
      raise=1mm,
      text={|\ttfamily\scriptsize|next},
      text align={align=center},
    }
  }
]

\tikzstyle{with reverse next label}=[
  preaction={
    decorate,
    decoration={
      raise=1mm,
      reverse path,
      text color=gray!75,
      text along path,
      text={|\ttfamily\scriptsize|next},
      text align={align=center},
    }
  }
]

\tikzstyle{plus step}=[
  postaction={
    decorate,
    decoration={
      raise=1mm,
      text along path,
      text align=center,
      text={|\ttfamily \scriptsize|+ step}
    }
  }
]

\tikzstyle{internal operation}=[
  text=black!75,
  scale=0.9,
]

\tikzstyle{predicate evaluation}=[
  node distance=.25,
  text=black!75,
  scale=0.8,
]

\tikzstyle{predicate evaluation edge}=[
  draw=black!50,
  line width=0.1mm,
  decorate,
  decoration={
    snake,
    amplitude=.5mm,
    segment length=2mm,
  }
]

\tikzstyle{inverse measure}=[
  arrows={Triangle[angle=15:3mm 1]-}
]
\tikzstyle{double measure}=[
  arrows={Triangle[angle=15:3mm 1]-Triangle[angle=15:3mm 1]}
]

\newcommand{\measure}[4][1ex] {
  \begin{scope}
    \tikzstyle{every path}=[line width=0.1mm]
    \draw (#3) -- ++(0, #1) +(0, 1.5mm) -- +(0, -1.5mm);
    \draw (#4) -- ++(0, #1) +(0, 1.5mm) -- +(0, -1.5mm);
    \draw ($ (#3) + (0, #1) $) [double measure] -- node [above] {#2} ($ (#4) + (0, #1) $);
  \end{scope}
}
\newcommand{\rightmeasure}[4][1ex] {
  \begin{scope}
    \tikzstyle{every path}=[line width=0.1mm]
    \draw (#3) -- ++(0, #1) +(0, 1.5mm) -- +(0, -1.5mm);
    \draw (#4) -- ++(0, #1) +(0, 1.5mm) -- +(0, -1.5mm);
    \draw (#3) ++(0, #1) [inverse measure] -- +(-6mm, 0);
    \draw (#4) ++(0, #1) [inverse measure] -- node [above, pos=.6] {#2} +(12mm, 0);
  \end{scope}
}

\tikzstyle{brace}=[
  decorate,
  decoration={
    brace,
    mirror,
    raise=2mm,
    amplitude=2mm,
  },
  line width=0.2mm,
]

\newcommand{\topbrace}[3][lastbracepoint] {
  \draw ([xshift=0.5mm] #2) [brace, decoration={mirror=false}] -- ([xshift=-0.5mm] #3);
  \coordinate (#1) at ($ ($ (#2)!.5!(#3) $) + (0, 4mm) $);
}

\newcommand{\bottombrace}[3][lastbracepoint] {
  \draw ([xshift=0.5mm] #2) [brace] -- ([xshift=-0.5mm] #3);
  \coordinate (#1) at ($ ($ (#2)!.5!(#3) $) - (0, 4mm) $);
}

\newcommand{\toppointer}[2]{
  \draw (#1) [latex-] -- ($ (#1) + (0, 1) $) node [above] {#2};
}

\newcommand{\param}[1] {
  \texttt{#1}
}

\newcommand{\subparam}[2] {
  \texttt{#1}_{\texttt{\scriptsize {#2}}}
}
 
\newcommand{\topparam}[2]{
  \node [arrow box, draw, arrow box arrows={south:2.5mm}, anchor=south] at ([yshift=1mm] #1) {#2};
}
\newcommand{\bottomparam}[2]{
  \node [arrow box, draw, arrow box arrows={north:2.5mm}, anchor=north] at ([yshift=-1mm] #1) {#2};
}
\newcommand{\leftparam}[2]{
  \node [arrow box, draw, arrow box arrows={east:2.5mm}, anchor=east] at ([xshift=-1mm] #1) {#2};
}
\newcommand{\rightparam}[2]{
  \node [arrow box, draw, arrow box arrows={west:2.5mm}, anchor=west] at ([xshift=1mm] #1) {#2};
}



\matrix (A) [collection] {
  \node (a1)   {$a_1$};     &
  \ellipsis                 &
  \node (an-i) {$a_{n-i}$}; &
  \ellipsis                 &
  \node (an-2) {$a_{n-2}$}; &
  \node (an-1) {$a_{n-1}$}; &
  \node (an)   {$a_n$};     \\
};

\matrix (B) [collection, below=3cm of A, matrix anchor=bn.east] at (an.east) {
  \node (bn-i) {$a_{n-i}$}; &
  \ellipsis                 &
  \node (bn-2) {$a_{n-2}$}; &
  \node (bn-1) {$b$}; &
  \node (bn)   {$a_n$};     \\
};

\node [draw, ellipse callout, callout absolute pointer={([xshift=-1mm] B.west)}] at ([xshift=-1cm] B.west) {$\param{that}$};
\draw (an-i.south east) [draw=none] -- node {\trueseq} (bn-2.north west);
\draw ([yshift=5mm] A.north east) [dashed] -- ([yshift=-5mm] B.south east);

\begin{scope}
  \tikzstyle{every path}=[Latex-Latex]
  \tikzstyle{every node}=[arrow label]
  \draw (an-i) -- node (t1) {=? \true} (bn-i);
  \draw (an-2) -- node (t2) {=? \true} (bn-2);
  \draw (an-1) -- node (t3) {=? \false} (bn-1);
\end{scope}

\node (B) [right=3cm of t3] {\false};
\draw [arrow] (t3) -- (B);

\draw (t1.north west) [rounded corners=0.5em] rectangle (t3.south east);

    \drawbg

    \end{tikzpicture}
\end{document}

